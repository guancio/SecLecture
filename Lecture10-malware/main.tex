\documentclass{beamer}
\usetheme{Frankfurt}

\usepackage{listings}

\newcommand{\todo}[1]{\alert{TODO #1}}

\title{Malware}
\subtitle{Lecture 10 \\ Computer Security DD2395}
\author[R. Guanciale]{
  Roberto Guanciale\\
  robertog@kth.se
}
\date{2013-11-04}
\begin{document}

\begin{frame}[plain]
  \titlepage
\end{frame}

\begin{frame}{Lab W}
  \begin{itemize}
  \item Mess
  \end{itemize}
\end{frame}

\begin{frame}{Malicious Software}
  \begin{itemize}
  \item programs exploiting system vulnerabilities 
  \item  known as malicious software or malware 
    \begin{itemize}
    \item  program fragments that need a host program  
    \end{itemize}
  \item  e.g. viruses, logic bombs, and backdoors 
    \begin{itemize}
    \item  independent self-contained programs 
    \end{itemize}
  \item  e.g. worms, bots 
    \begin{itemize}
    \item  replicating or not 
    \end{itemize}
  \item  sophisticated threat to computer systems 
  \end{itemize}
\end{frame}


\begin{frame}{Malware Terminology}
  \begin{itemize}
  \item  Virus 
  \item  Worm 
  \item  Logic bomb 
  \item  Trojan horse 
  \item  Backdoor (trapdoor)‏ 
  \item  Mobile code 
  \item  Auto-rooter Kit (virus generator)‏ 
  \item  Spammer and Flooder programs 
  \item  Keyloggers, Spyware 
  \item  Rootkit 
  \item  Zombie, bot 
  \item  Adware   
  \end{itemize}
\end{frame}
 

\begin{frame}{Would you trust this program?}
  \includegraphics[width=0.8\linewidth]{freeantispy}
\end{frame}

\begin{frame}{Trojan Horse}
  \begin{itemize}
  \item  First identified at NSA in 1972 by Daniel 
    Edwards 
  \item  It's a program with two purposes, one obvious 
    and one hidden from the user 
  \item  Today it's often used to install other software or 
    backdoors 
  \item  Trojan horses can be built from existing 
    programs using a special wrapper 
  \item  Or designed from the start to be one.
  \end{itemize}
\end{frame}

\begin{frame}{What would you do?}
  \begin{itemize}
  \item  How to get someone to run a trojan? 
  \item  How to not run a trojan?
  \end{itemize}
\end{frame}

\begin{frame}{Backdoor}
  \begin{itemize}
  \item  Software that gives access to a system 
  \item  Bypassing OS restrictions 
  \item  Can be part of a trojan 
  \item  Often installed for legitimate reasons 
  \item  Only to later be abused 
  \item  Typically very very hard to find
  \end{itemize}
\end{frame}

\begin{frame}{Legitimate Reasons?}
  \begin{itemize}
  \item  What would be a legitimate reason to install a 
backdoor?  
  \end{itemize}
\end{frame}
 
\begin{frame}{Grayware}
  \begin{itemize}
  \item  In the gray zone between harmless and 
    harmful, mostly annoying 
  \item  Popup windows 
  \item  For teh lulz 
  \item  Can include adware, spyware
  \end{itemize}
\end{frame}

\begin{frame}{Logic Bomb }
  \begin{itemize}
  \item  A small bit of code that triggers on a specific 
    condition 
  \item  Typically with malicious results 
  \item  No vector for spreading 
  \item  Installed directly
  \end{itemize}
\end{frame}

\begin{frame}{Viruses }
  \begin{itemize}
  \item  piece of software that infects programs 
    \begin{itemize}
    \item  modifying them to include a copy of the virus 
    \item  so it executes secretly when host program is run 
    \end{itemize}
  \item  specific to operating system and hardware 
    \begin{itemize}
    \item  taking advantage of their details and weaknesses 
    \end{itemize}
  \item  a typical virus goes through phases of: 
    \begin{itemize}
    \item  dormant 
    \item  propagation 
    \item  triggering 
    \item  execution
    \end{itemize}
  \end{itemize}
\end{frame}

\begin{frame}{Virus Structure }
  \begin{itemize}
  \item  components: 
    \begin{itemize}
    \item  infection mechanism - enables replication 
    \item  modification engine - for disguise 
    \item  trigger - event that makes payload activate 
    \item  payload - what it does, malicious or benign 
    \end{itemize}
  \item  prepended / appended / embedded 
  \item  when infected program invoked, executes 
    virus code then original program code 
  \item  can block initial infection (difficult)‏ 
  \item  or propagation (with access controls)
  \end{itemize}
\end{frame}

\begin{frame}{Virus Structure }
  \begin{itemize}
  \item  components: 
    \begin{itemize}
    \item  infection mechanism - enables replication 
    \item  modification engine - for disguise 
    \item  trigger - event that makes payload activate 
    \item  payload - what it does, malicious or benign 
    \end{itemize}
  \item  prepended / appended / embedded 
  \item  when infected program invoked, executes 
    virus code then original program code 
  \item  can block initial infection (difficult)‏ 
  \item  or propagation (with access controls)
  \end{itemize}
\end{frame}

\begin{frame}[fragile]{Naive virus structure}
  \begin{verbatim}
0x00000000 goto main;
0x00000004 666;
main:      infect();
           if trigger() then do-damage();
           goto victim:
infect:    file = get-random-exec();
           if file[0x00000004] = 666 goto infect;
           prepend V to file;
           return;
do-damage: ...
trigger:   ...
victim:    ...
  \end{verbatim}
\end{frame}


\begin{frame}{Virus Classification}
  \begin{itemize}
  \item  boot sector 
  \item  file infector 
  \item  macro virus 
  \item  encrypted virus: different keys 
  \item  stealth virus: evade detection, e.g. 
    compression 
  \item  polymorphic virus 
  \item  metamorphic virus
  \end{itemize}
\end{frame}

\begin{frame}[fragile]{Compression virus}
  \begin{verbatim}
0x00000000 goto main;
0x00000004 666;
main:      infect();
           if trigger() then do-damage();
           uncompressVictim();
           goto victim:
infect:    file = get-random-exec();
           if file[0x00000004] = 666 goto infect;
           compress file;
           prepend V to file;
           return;
do-damage: ...
trigger:   ...
victim:    ...
  \end{verbatim}
\end{frame}

\begin{frame}{Compression virus}
\includegraphics[width=0.8\linewidth]{compressingVirus}
\end{frame}

\begin{frame}{Network}
\includegraphics[width=0.8\linewidth]{network}
\end{frame}

\begin{frame}{Polymorphic Virus }
  \begin{itemize}
  \item  A virus can take things one step further: 
    Rebuild the whole virus at every infection to 
    something functionally identical 
  \item  There are many ways to do nothing on a 
    computer 
  \item  Instructions can be reordered in many ways 
  \item  To detect these the AV engine often has to 
    simulate the virus to figure out what it is.
  \end{itemize}
\end{frame}

\begin{frame}{Metamorphic Virus}
  \begin{itemize}
  \item  Complete rewrite 
  \item  Can also change behavior
  \end{itemize}
\end{frame}
 
\begin{frame}{Macro Virus }
  \begin{itemize}
  \item  became very common in mid-1990s since 
    \begin{itemize}
    \item  platform independent 
    \item  infects documents 
    \item  is easily spread 
    \end{itemize}
  \item  exploit macro capability of office apps 
    \begin{itemize}
    \item  executable program embedded in office doc 
    \item  often a form of Basic 
    \end{itemize}
  \item  more recent releases include protection 
  \item  recognized by many anti-virus programs 
  \end{itemize}
\end{frame}

\begin{frame}{E-Mail Viruses }
  \begin{itemize}
  \item  more recent development 
  \item  e.g. Melissa 
    \begin{itemize}
    \item  exploits MS Word macro in attached doc 
    \item  if attachment opened, macro activates 
    \item  sends email to all on users address list 
    \item  and does local damage 
    \end{itemize}
  \item  then saw versions triggered reading email 
  \item  hence much faster propagation
  \end{itemize}
\end{frame}

\begin{frame}{Virus Countermeasures}
  \begin{itemize}
  \item  prevention - ideal solution but difficult 
  \item  realistically need: 
    \begin{itemize}
    \item  detection 
    \item  identification 
    \item  removal 
    \end{itemize}
  \item  if detected but can't identify or remove, must 
    discard and replace infected program
  \end{itemize}
\end{frame}

\begin{frame}{Anti-Virus Evolution }
  \begin{itemize}
  \item  virus \& antivirus tech have both evolved 
  \item  early viruses simple code, easily removed 
  \item  as become more complex, so must the 
    countermeasures 
  \item  generations 
    \begin{itemize}
    \item  first - signature scanners 
    \item  second - heuristics 
    \item  third - identify actions 
    \item  fourth - combination packages 
    \end{itemize}
  \end{itemize}
\end{frame}

\begin{frame}{Generic Decryption }
  \begin{itemize}
  \item  runs executable files through GD scanner: 
    \begin{itemize}
    \item  CPU emulator to interpret instructions 
    \item  virus scanner to check known virus signatures 
    \item  emulation control module to manage process 
    \end{itemize}
  \item  lets virus decrypt itself in interpreter 
  \item  periodically scan for virus signatures 
  \item  issue is long to interpret and scan 
    \begin{itemize}
    \item  tradeoff chance of detection vs time delay
    \end{itemize}
  \item virtualization
  \end{itemize}
\end{frame}

\begin{frame}{Digital Immune Systems}
\includegraphics[width=0.8\linewidth]{digitalImmuneSystem}
\end{frame}

\begin{frame}{Behavior-Blocking Software}
\includegraphics[width=0.8\linewidth]{behaviorBlocking}
\end{frame}

\begin{frame}{Worms}
  \begin{itemize}
  \item  replicating program that propagates over net 
    \begin{itemize}
    \item  using email, remote exec, remote login 
    \end{itemize}
  \item  has phases like a virus: 
    \begin{itemize}
    \item  dormant, propagation, triggering, execution 
    \item  propagation phase: searches for other systems, connects to 
      it, copies self to it and runs 
    \end{itemize}
  \item  may disguise itself as a system process 
  \item  implemented by Xerox Palo Alto labs in 1980's
  \end{itemize}
\end{frame}

\begin{frame}{Morris Worm}
  \begin{itemize}
  \item  one of best known early worms 
  \item  released by Robert Morris in 1988 
  \item  various attacks on UNIX systems 
    \begin{itemize}
    \item  cracking password file to use login/password to 
      logon to other systems 
    \item  exploiting a bug in the finger protocol 
    \item  exploiting a bug in sendmail 
    \end{itemize}
  \item  if succeed have remote shell access 
    \begin{itemize}
    \item  sent bootstrap program to copy worm over
    \end{itemize}
  \end{itemize}
\end{frame}

\begin{frame}{Worm Propagation Model}
\includegraphics[width=0.8\linewidth]{wormPropagation}
\end{frame}


\begin{frame}{Recent Worm Attacks}
  \begin{itemize}
  \item  Code Red 
    \begin{itemize}
    \item  July 2001 exploiting MS IIS bug 
    \item  probes random IP address, does DDoS attack 
    \item  consumes significant net capacity when active 
    \end{itemize}
  \item  Code Red II variant includes backdoor 
  \item  SQL Slammer 
    \begin{itemize}
    \item  early 2003, attacks MS SQL Server 
    \item  compact and very rapid spread 
    \end{itemize}
  \item  Mydoom 
    \begin{itemize}
    \item  mass-mailing e-mail worm that appeared in 2004 
    \item  installed remote access backdoor in infected systems 
    \end{itemize}
  \end{itemize}
\end{frame}

\begin{frame}{Recent Worm Attacks}
  \begin{itemize}
  \item  Conficker 2009 
  \item  Stuxnet 2010 
  \item  Duqu 2011 
  \item  Flame 2012
  \end{itemize}
\end{frame}

\begin{frame}{Worm Technology}
  \begin{itemize}
  \item  multiplatform 
  \item  multi-exploit 
  \item  ultrafast spreading 
  \item  polymorphic 
  \item  metamorphic 
  \item  transport vehicles 
  \item  zero-day exploit
  \end{itemize}
\end{frame}
 
\begin{frame}{Worm Countermeasures}
  \begin{itemize}
  \item  overlaps with anti-virus techniques 
  \item  once worm on system A/V can detect 
  \item  worms also cause significant net activity 
  \item  worm defense approaches include: 
    \begin{itemize}
    \item  signature-based worm scan filtering 
    \item  filter-based worm containment 
    \item  payload-classification-based worm containment 
    \item  threshold random walk scan detection 
    \item  rate limiting and rate halting 
    \end{itemize}
  \end{itemize}
\end{frame}

\begin{frame}{Proactive Worm Containment}
\includegraphics[width=0.8\linewidth]{wormContaiment}
\end{frame}

\begin{frame}{Network Based Worm Defense}
\includegraphics[width=0.8\linewidth]{networkBasedAntiWorm}
\end{frame}


\begin{frame}{Bots}
  \begin{itemize}
  \item  program taking over other computers 
  \item  to launch hard to trace attacks 
  \item  if coordinated form a botnet 
  \item  characteristics: 
    \begin{itemize}
    \item  remote control facility 
    \end{itemize}
  \item  via IRC/HTTP etc 
    \begin{itemize}
    \item  spreading mechanism 
    \end{itemize}
  \item  attack software, vulnerability, scanning strategy 
  \item  various counter-measures applicable 
  \end{itemize}
\end{frame}

\begin{frame}{Rootkits}
  \begin{itemize}
  \item  set of programs installed for admin access 
  \item  malicious and stealthy changes to host O/S 
  \item  may hide its existence 
    \begin{itemize}
    \item  subverting report mechanisms on processes, files, registry entries 
      etc 
    \end{itemize}
  \item  may be: 
    \begin{itemize}
    \item  persistent or memory-based 
    \item  user or kernel mode 
    \end{itemize}
  \item  installed by user via trojan or intruder on system 
  \item  range of countermeasures needed
  \end{itemize}
\end{frame}


\begin{frame}{Rootkit System Table Mods}
\includegraphics[width=0.8\linewidth]{rootkit}
\end{frame}
 
\begin{frame}{Summary}
  \begin{itemize}
  \item  introduced types of malicous software 
    \begin{itemize}
    \item  incl backdoor, logic bomb, trojan horse 
    \item  virus types and countermeasures 
    \end{itemize}
  \item  worm types and countermeasures 
  \item  bots 
  \item  rootkits
  \end{itemize}
\end{frame}


\end{document}




%%% Local Variables: 
%%% mode: latex
%%% TeX-master: t
%%% End: 
