\documentclass{beamer}
\usetheme{Frankfurt}

\usepackage{listings}

\newcommand{\todo}[1]{\alert{TODO #1}}

\title{Malware}
\subtitle{Lecture 10 \\ Computer Security DD2395}
\author[R. Guanciale]{
  Roberto Guanciale\\
  robertog@kth.se
}
\date{2013-11-04}
\begin{document}

\begin{frame}{SMTP}
  \begin{itemize}
  \item A sends: from: c@C to: b@B hello
  \item B sends: from: b@B to: c@C vacation message
  \item C sends: from: c@C to: b@B vacation message
  \item Quota full
  \item User not allowed e.g. SMTP fax proxy
  \item Non existing user
  \end{itemize}
\end{frame}

\begin{frame}{ZIP}
  \begin{itemize}
  \item Infinite recursive zip: e.g. r.zip
  \item Bombs (e.g. non recursive files)    
    \begin{itemize}
      \item gzip: 100 GB, 97 MB compressed, 1000:1 ration
      \item bzip2: 100 GB, 69 KB, $1.6*10^6$:1
      \item PNG image: 19000 x 19000, 1-bit (45 MB) expand in 24-bit
        color to 1 GB, 44 KB compressed, 1000:1
    \end{itemize}
  \end{itemize}
\end{frame}

\begin{frame}[fragile]{XML}
  \footnotesize \begin{verbatim}
<?xml version="1.0"?>
<!DOCTYPE lolz [
 <!ENTITY lol "lol">
 <!ELEMENT lolz (#PCDATA)>
 <!ENTITY lol1 "&lol;&lol;&lol;&lol;&lol;&lol;&lol;&lol;&lol;&lol;">
 <!ENTITY lol2 "&lol1;&lol1;&lol1;&lol1;&lol1;&lol1;&lol1;&lol1;&lol1;&lol1;">
 <!ENTITY lol3 "&lol2;&lol2;&lol2;&lol2;&lol2;&lol2;&lol2;&lol2;&lol2;&lol2;">
 <!ENTITY lol4 "&lol3;&lol3;&lol3;&lol3;&lol3;&lol3;&lol3;&lol3;&lol3;&lol3;">
 <!ENTITY lol5 "&lol4;&lol4;&lol4;&lol4;&lol4;&lol4;&lol4;&lol4;&lol4;&lol4;">
 <!ENTITY lol6 "&lol5;&lol5;&lol5;&lol5;&lol5;&lol5;&lol5;&lol5;&lol5;&lol5;">
 <!ENTITY lol7 "&lol6;&lol6;&lol6;&lol6;&lol6;&lol6;&lol6;&lol6;&lol6;&lol6;">
 <!ENTITY lol8 "&lol7;&lol7;&lol7;&lol7;&lol7;&lol7;&lol7;&lol7;&lol7;&lol7;">
 <!ENTITY lol9 "&lol8;&lol8;&lol8;&lol8;&lol8;&lol8;&lol8;&lol8;&lol8;&lol8;">
]>
<lolz>&lol9;</lolz>
  \end{verbatim}
  \begin{itemize}
    \item 1 KB file
    \item 3GB XML (DOM worst)
  \end{itemize}
\end{frame}

\begin{frame}{Fork Bomb}
  \begin{itemize}
  \item PostgreSQL 7.2
  \item Uses elapsed milliseconds to compute number of thread to spawn
  \item No limit check
  \item<2-> What happen if you advance the date of one year?
  \item<3-> Network Time Protocol attack
 \end{itemize}
\end{frame}

\begin{frame}{Old SMS attack}
  \begin{itemize}
  \item Limited number of SMSs (i.e. 9) stored into the phone
  \item No limit in the reception from the provider
  \item Usage of zombie and Internet free SMS servers
  \item Fill and block the SMS reception of a user
 \end{itemize}
\end{frame}

\begin{frame}{HTTP}
  \begin{itemize}
  \item Distributed flood (spiders)
  \item Slashdotted
  \item Slowloris
  \begin{itemize}
    \item HTTP server: one thread/one request
    \item Open connection
    \item Infinitely (and slowly) send HTTP headers
    \item Consume the thread pool
  \end{itemize}
 \end{itemize}
\end{frame}

\end{document}




%%% Local Variables: 
%%% mode: latex
%%% TeX-master: t
%%% End: 
