\documentclass{beamer}
\usetheme{Frankfurt}

\usepackage{listings}

\title{Social engeneering and Secure Multiparty Computation}
\subtitle{Lecture 12 \\ Computer Security DD2395}
\author[R. Guanciale]{
  Roberto Guanciale\\
  robertog@kth.se
}
\date{2013-12-17}
\begin{document}

\begin{frame}[plain]
  \titlepage
\end{frame}

\begin{frame}{Human resource security}
  \begin{itemize}
    \item Security of a system depends on the user behavior
    \item e.g. DB with 135 M users, top 100 passwords cover 6 M users
    \item e.g. users sharing passwords
    \item e.g. unattended workstations
    \item Users can represent information flows that are not
      constrained by software/hardware
    \item Business process and procedures
    \item These information can carry authentication tokens
  \end{itemize}
\end{frame}


\begin{frame}{How Social Engineers Work}
  
  \begin{quote}
The social engineer employs the same persuasive 
techniques the rest of us use every day. We take 
on roles. We try to build credibility. We call in 
reciprocal obligations. But the social engineer 
applies these techniques in a manipulative, 
deceptive, highly unethical manner, often to 
devastating effect.
  \end{quote}
Brad Sagan, social psychologist
\end{frame}

\begin{frame}{What for?}
  \begin{itemize}
    \item Industrial spying 
    \item Access to resources 
    \item Data theft
    \item Identity theft
  \end{itemize}
\end{frame}
 

\begin{frame}{Exploits}
  \begin{itemize}
    \item The desire to help 
    \item Attribution
    \item Linking
    \item Fear
    \item Reactance (e.g. reverse psychology)
  \end{itemize}
\end{frame}

\begin{frame}{Techniques: Trapping of roles}
  \begin{itemize}
    \item Build a common enemy / being ally
    \item The devil-snake: genesis book, the snake convinces Eve that God was
      keeping his best power for himself, thus forbidding to eat from
      the ``Tree of Life''
    \item<2-> Today: call an enterprise and inform about the internal
      corporate restructuring
    \item<2-> The enterprise stakholders the common enemy 
  \end{itemize}
\end{frame}

\begin{frame}{Techniques: Read victim's openness/curiosity}
  \begin{itemize}
    \item Allow the victim to found a gift
    \item The trojan horse: Ulysses fooled the Trojans into believing
       his army had abandoned their siege by leaving a large
      wooden horse (full of soldiers) - a ``gift'' - outside the gates of the city
    \item<2-> Today: (Baiting) left a USB pen in the parking lot
    \item<2-> Today: left a DVD titled ``CVC recodring''
    \item<2-> Today: build an opened WiFi (at the airport) and dump the traffic
  \end{itemize}
\end{frame}

\begin{frame}{Techniques: offer benefits}
  \begin{itemize}
    \item Offer benefits to a victim to spread
    \item Ponzi scheme:  told friends if they invested with him, he
      would double their investment within 90 days. His scheme
      involved using new money to pay off older investors, and the
      entire business was being run at a loss
    \item Today: Bernie Madoff ($\$$ 65 billion, 10/30 years)
    \item<2-> Today: Bitcoins?
  \end{itemize}
\end{frame}

\begin{frame}{Techniques: Credibility}
  \begin{itemize}
    \item Look official, act official, people will believe you are official
    \item Frank Abagnale was able to
      convince Pan Am employees, and many others, that he was an
      airline pilot in the 1960, by dressing in a Pan Am pilot's uniform
    \item<2-> Today: (Attribution) during a phone call demostrate to know
      the organizational chart, to persuade the listener that you are
      an organizazion employee
    \item<2-> Today: (Attribution) use the exact lingo
  \end{itemize}
\end{frame}


\begin{frame}{Techniques: Piggybacking}

  \begin{itemize}
    \item when a person tags along with
another person who is authorized to gain entry into a restricted area,
or pass a certain checkpoint.
    \item Following an individual authorized to enter
      a location, giving the appearance of being legitimately escorted
    \item Joining a large crowd authorized to enter, and pretending to
      be a member of the crowd that is largely unchecked
  \end{itemize}

\end{frame}


\begin{frame}{Techniques: Quid pro quo}
  
  \begin{itemize}
    \item An attacker calls random numbers at a company, claiming to
      be calling back from technical support. Eventually this person
      will hit someone with a legitimate problem, grateful that
      someone is calling back to help them. The attacker will ``help''
      solve the problem and, in the process, have the user type
      commands that give the attacker access or launch malware.
    \item Create then fix a problem
  \end{itemize}
\end{frame}

\begin{frame}{Other Techniques}
  \begin{itemize}
    \item Distracting from systematic thinking
    \item Momentum of compliance
    \item Hide questions
    \item Get pieces from different sources
    \item Test with personal information
    \item Phishing
  \end{itemize}
\end{frame}

\begin{frame}{Real case: the first call}
  \begin{itemize}
    \item[victim] Tom DeLay, Bookkeeping
    \item[attacker] Help Desk.
      We're trying to troubleshoot a computer networking problem.
    \item[\dots]
    \item[attacker] Let me give you my cell phone number.
      Then you can reach me directly if you need to.
    \item[victim] That'd be great. Go ahead
    \item[\dots]
    \item[attacker] I need to check which port your computer is
      connected to. Take a look on your computer and see if there's a
      sticker somewhere that says something like ``Port Number''.
    \item Let Me Help You
    \item Disclosing a non-sensitive information
  \end{itemize}
\end{frame}

\begin{frame}{Real case: the second call, two days later}
  \begin{itemize}
    \item[IT Guy] Hi, this is Bob
    \item[attacker] I'm in Tom DeLay's office in
      Bookkeeping. We're trying to troubleshoot a cabling problem. I
      need you to disable Port 6-47.
    \item[\dots]
    \item Tom DeLay's office in Bookkeeping, Port 6-47
    \item Attribution, Lingo
  \end{itemize}
\end{frame}

\begin{frame}{Real case: the third call, INCOMING}
  \begin{itemize}
    \item[victim] Hi, I've a network problem as you said, please
      help me, I've a deadline
    \item[attacker] I can solve the problem in 3 hours
    \item[victim] Can you do faster
    \item[attacker] I give you the maximum priority, I will try to fix
      as soon as possible
    \item[\dots] (call to the IT to restore the network)
    \item Became an Ally
  \end{itemize}
\end{frame}

\begin{frame}{Real case: the forth call}
  \begin{itemize}
    \item[attacker] Can you check the network?
    \item[victim] Great it works!!!
    \item[attacker] We still have firewall problems, if you want you
      can install this software to prevent your network from being
      isolated during our reconfigurations 
    \item[victim] Thanks!!
    \item[\dots] 
    \item Authentication
    \item Take into account what can be done today with VoIP systems
  \end{itemize}
\end{frame}

\begin{frame}{Anti-spam systems}
  \begin{itemize}
    \item CAPTCHA (Completely Automated Public Turing test to tell
      Computers and Humans Apart)
    \item In October 2013, artificial intelligence company Vicarious
      claimed that it had developed software that was able to solve
      modern CAPTCHAs with character recognition rates of up to $90\%$
    \item<2-> Hire People To Solve CAPTCHA Challenges: 
set up a ``Turing Farm'' to
employ people to look at computer screens and answer CAPTCHA
challenges. 
\item<3-> ``Turing Porn/Warez Farm'', in which spammers also run a porn site that is
  gated by CAPTCHA challenges, which are actually ripped directly from
  the target service CAPTCHA
  \end{itemize}
\end{frame}

\begin{frame}{Phases}
  \begin{itemize}
    \item Pretexting
    \item Get data
    \item Keep connections
    \item Combine data
    \item Use it
  \end{itemize}
\end{frame}

\begin{frame}{Countermeasure}
  \begin{itemize}
    \item Clear protocols that are enforced
    \item Awareness training
    \item Simple rules to define sensitive information
    \item Simple rule to identify / authorize
    \item Data classification policy
    \item Testing
    \item Least Privilege
    \item Open Design
  \end{itemize}
\end{frame}

\begin{frame}{Secure Multiparty Computation}
  \begin{itemize}
    \item $n$ participants with their private inputs
      $x_1,\dots,x_n$
    \item compute $f_i(x_1,\dots,x_n)$ without disclosing the private
      inputs
    \item e.g. the millionaire problem:
      $x_1 > x_2$
    \item e.g. social networks
    \item e.g. enterprise cooperation
  \end{itemize}
\end{frame}

\begin{frame}{Building blocks (1): Oblivious transfer}
  \begin{itemize}
    \item $1-2$ OT
    \item $x_1 = \{m_0, m_1\}$, $x_2 = \{0, 1\}$
    \item $f_2(x_1, x_2) = x_1[x_2] = m_{x_2}$, $f_1(x_1, x_2) = 1$
      
      \begin{enumerate}
        \item $1$ generates private/public key $p_k, P_k$, generates random $r_0, r_1$
        \item $1 \rightarrow 2: P_k,r_0,r_1$
        \item $2$ generates random $k$
        \item $2 \rightarrow 1: P_k(k) + r_{x_2} = v$
        \item $1$ computes $k_i = p_k(v-r_i)$: $k_0=k$ or $k_1=k$
        \item $1 \rightarrow 2: m'_i = m_i+k_i$
        \item $2$ computes $m'_{x_2} - k$
      \end{enumerate}
    \item Can be extended as $1$ out of $n$ transfer
  \end{itemize}
\end{frame}


\begin{frame}{Building blocks (2): Yao's garbled circuits}
  
  \begin{itemize}
  \item let $g$ be a boolean gate $g:\{0,1\} \times \{0,1\}
    \rightarrow \{0,1\}$ (and, or, xor, \dots)
  \item let $k^0_1$, $k^1_1$, $k^0_2$, $k^1_2$, $k^0_3$, $k^1_3$ be
    six keys of a symmetric encryption scheme
  \item encode the gateway as a random permutation of
    $c_{i,j} = E_{k^i_1}(E_{k^i_2}(k^{g(i,j)}_3))$
  \item executing a gate:
    
    \begin{itemize}
    \item having the gate encoding, and the cryptos representing its
      inputs $k'_1$ and $k'_2$
    \item the crypto output is the $k'_3=D_{k'_1}(D_{k'_2}(c_{i,j}))$
      such that $k'_3 \neq \bullet$
    \end{itemize}
  \end{itemize}
\end{frame}


% \begin{frame}{Garbled gate example}
%   \begin{tabular}{cccccccc}
%     1 & 2 & AND & OR & > & = \\
%     0 & 0 & 0   & 0  & 0 & 1 \\
%     0 & 1 & 0   & 1  & 0 & 0 \\
%     1 & 0 & 0   & 1  & 1 & 0 \\
%     1 & 1 & 1   & 1  & 0 & 1 \\
%   \end{tabular}
% \end{frame}

\begin{frame}{Building blocks (2): Yao's garbled circuits (2)}
  \begin{itemize}
  \item Let $m$ be a circuit: wires and gates
  \item names the wire $w_i$ such that
  \begin{itemize}
    \item if two wires exit from the same gate, they have the same name
  \end{itemize}
  \item for each wire, generates the keys $k^0_{w_i}$ and $k^1_{w_i}$
  \item generate the gate encoding
  \item the output gates have encoding $c_{i,j} =
    E_{k^i_1}(E_{k^i_2}(g(i,j)))$
  \end{itemize}
\end{frame}

\begin{frame}{Building blocks (2): Yao's garbled circuits (3)}
  \begin{itemize}
  \item Participant $1$ generates the keys and the circuit
  \item Participant $1$ sends the circuit to $2$
  \item Participant $1$ sends the encryption of its input bits to $2$
  \item How can the participant $2$ obtain the encryption of its input
    bits?
  \item<2-> OT-transfer, for each bit it executes the 1-2 OT to get
    the wire key corresponding to $0$ or $1$
  \item<3-> Participant $2$ executes the circuit
  \end{itemize}
\end{frame}


\begin{frame}{Exam}
  \begin{itemize}
  \item Only topics of the book
  \item Similar to the previous years
  \item Closed book
  \item Examples
    \begin{itemize}
      \item Explain the concepts of honeypot
      \item Million user names and passwords were leaked
        by hackers and put up on a web forum.
        The passwords were hashed but not salted
      \item Explain the strengths and weaknesses of each of the
        following firewall deployment scenarios: network perimeter,
        host based
      \item Explain the IP spoofing and its security effects
      \item Write a C program with a buffer overflow
      \item Describe the strategies used by a virus to hide himself
      \item Describe an cryptography algorithm
    \end{itemize}
  \end{itemize}
\end{frame}

% \begin{frame}{Example: $>$}
%   $max([b_{10}, b_{11}], [b_{20}, b_{21}] = (b_{10} > b_{20}) or
%   ((b_{10} = b_{20}) and (b_{11} > b_{22}))$
% \end{frame}

\end{document}

%%% Local Variables: 
%%% mode: latex
%%% TeX-master: t
%%% End: 
