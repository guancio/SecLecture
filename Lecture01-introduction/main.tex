\documentclass{beamer}
\usetheme{Frankfurt}

\usepackage{listings}

\title{Introduction}
\subtitle{Lecture 1 \\ Computer Security DD2395}
\author[R. Guanciale]{
  Roberto Guanciale\\
  robertog@kth.se
}
\date{2013-10-03}
\begin{document}

\begin{frame}[plain]
  \titlepage
\end{frame}

\begin{frame}{Outline for Today}
  \begin{itemize}
    \item About me
    \item About the course
    \item About you
    \item About computer security
  \end{itemize}
\end{frame}

\begin{frame}{About Roberto Guanciale}
  \begin{itemize}
    \item PostDoc on formal verification
    \item Enthusiast software developer
    \item Proven experience in designing and developing \alert{non}-secure
      systems
    \item robertog@kth.se
    \item \alert{Always} use the e-mail subject: [DD2395] ...
    \item Office: level 4, room 4424
    \item You are welcomed (no booking required, no fee) Wednesday:
      14:00-15:00
  \end{itemize}
\end{frame}

\begin{frame}{About the course}
  Lectures and Labs
  \begin{itemize}
    \item From now until December
    \item Ca. 3 ECTS in P1 and 3 ECTS in P2
    \item Labs start September 18
  \end{itemize}
\end{frame}

\begin{frame}{General Goals}
  \begin{itemize}
    \item Learn about security concepts
    \item Have tools and methods to reason about security
    \item Spot threats, vulnerabilities
    \item Know and propose counter-measures
    \item Present concepts to others 
  \end{itemize}
\end{frame}

\begin{frame}{Learning Outcomes}
  You should be able to
  \begin{itemize}
    \item recognize threats to confidentiality, integrity, and
      availability of systems
    \item explain the basic computer security terminology and concepts
      and use them correctly 
    \item find and apply documentation of security-related problems
      and tools 
    \item analyze code or system descriptions in terms
      of their security
    \item identify vulnerabilities of such code or descriptions and
      predict their corresponding threats
    \item compare counter-measures and evaluate their side-effects
    \item present and explain their reasoning to others
    \item exploit vulnerabilities of code or systems
  \end{itemize}
\end{frame}

\begin{frame}{People}
  \begin{itemize}
    \item Course leader/exams: Roberto Guanciale
    \item Former Course leader/exams: Sonja Buchegger (KTH professor)
    \item Invited speakers:
      Alexander Baltatzis,
      Olof Hagsand,
      Gunnar Karlsson,
      Mikael,
      OWASP,
      Marten Trolin 
    \item Lab assistants: 
      Benjamin Greschbach,
      Guillermo Rodriguez Cano
  \end{itemize}
\end{frame}

\begin{frame}{Course info and breaking news}
  \begin{itemize}
    \item Check course website regularly for updates!
    \item DD2395 dasak12
    \item http://www.csc.kth.se/utbildning/kth/kurser/DD2395/dasak12/
  \end{itemize}
\end{frame}

\begin{frame}{Times and Places}
  \begin{itemize}
    \item look at schema, course code DD2395
  \end{itemize}
\end{frame}

\begin{frame}{Lectures Content}
  \begin{tabular}{cl|cl}
    01 & Admin, Intro [Ch.1] &
    02 & Operating systems \\
    03 & Computer networking &
    04 & Cryptography [2,20] \\
    05 & Authentication [3] &
    06 & Web security \\
    07 & Access control [4] &
    08 & Intrusion detection [8] \\
    09 & Firewalls [9] &
    10 & Malware [7] \\
    11 & Denial of Service [8] &
    12 & Multi-level security [10] \\
    13 & Audits &
    14 & Buffer overflows [11] \\
    15 & Software engineering &
    16 & Social engineering
  \end{tabular}
\end{frame}

\begin{frame}{Lab Exercises}
  \begin{itemize}
    \item See schema for times and rooms
    \item Instructions on web site on September 11
    \item 4 different exercises
      \begin{enumerate}
        \item on GnuPG, remote or at CSC
        \item on iptables/firewalls, at CSC
        \item on web attacks, remote or at CSC
        \item presentation at CSC, report, assess
      \end{enumerate}
  \end{itemize}
\end{frame}

\begin{frame}{Seminar}
  \begin{itemize}
    \item Presentation and demo on computer security 
      topic in a seminar
    \item Groups of 2-3 students
    \item Group seminars, schedule in schema, 
      signup on course website
  \end{itemize}
\end{frame}

\begin{frame}{Deadlines}
  Fridays, at 5pm 
  \begin{itemize}
  \item 28/9 GPG lab completion 
  \item 5/10 Iptables lab registration 
  \item 12/10 Seminar topic registration 
  \item 26/10 Web attacks lab completion 
  \item 2/11 Seminar time registration 
  \item 9/11 Seminar report completion 
  \item 16/11 Seminar feedback completion 
  \end{itemize}
\end{frame}

\begin{frame}{Exam}
  \begin{itemize}
  \item December 14, 2012
  \item \alert{Mandatory} registration 
  \item Re-exam in June 2013
  \end{itemize}
\end{frame}

\begin{frame}{Assessment, Grades}
  \begin{itemize}
  \item 6 ECTS in total, $\sim$ 160 hours of work 
  \item 3 ECTS Exam $\in [A \dots F]$
  \item 3 ECTS Labs $\in \{pass,fail\}$, no grades
  \item bonus points for exam (once) when bonus 
    requirements of lab fulfilled, see lab descriptions
  \end{itemize}
\end{frame}

\begin{frame}{Main didactic materials}
  \includegraphics[width=0.5\linewidth]{book01}
  \includegraphics[width=0.5\linewidth]{book01}
\end{frame}


\end{document}


%%% Local Variables: 
%%% mode: latex
%%% TeX-master: t
%%% End: 
